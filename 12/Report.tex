\documentclass[11pt]{article}
\usepackage{natbib,mybigpackage}
\usepackage{algorithm}
%\usepackage{program}
%\usepackage{algpseudocode}
\usepackage{algorithmic}
\usepackage{listings}

\usepackage{amssymb}

\usepackage{multirow}
%\def\xbf{\mathbf{x}}
%\def\zbf{\mathbf{z}}
%\def\xibf{\mathbf{\xi}}
\title{MA 226 - Assignment Report 12}
\author{Ayush Sharma\\150123046}
\begin{document}
\titlepage
\newpage
\begin{itemize}
\item[Q 1.] Generate the first 25 values of the Van der Corput sequence $x_1, x_2, \cdots, x_{25}$ using the radical inverse function $x_i \:= \phi_{2}(i)$ and list them in your report.
Next, generate the first 1000 values of this sequence and plot the overlapping pairs $(x_i, x_{i+1})$ as a two dimensional plot.
What do you observe?
Now, generate first 100 and 100000 values of this sequence and plot the sampled distributions for both the cases.
Compare these plots with the sampled distributions of 100 and 100000 values generated by an LCG, by plotting the sampled distributions in two graphs side by side for both the cases.
Specify the LCG that you have used.

\item[Q 2.] Generate the Halton sequence $x_i = (\phi_{2}(i), \phi_{3}(i))$ (as points in $\mathbb{R}^2$) and plot the first 100 and 100000 values.
What are your observations?\\
\textbf{Recall that the radical inverse function is defined by $\phi_{b}(i) = \sum_{k=0}^{j}d_{k}b^{-k-1}$ where $i = \sum_{k=0}^{j}d_{k}b^{k}$.}
\end{itemize}
\newpage
\noindent{\textbf{Code for R}}
\begin{lstlisting}
lcg <- function(a, b, m, seed, size) {
	x <- vector(length = size);
	x[1] = seed;
	for (i in 2:size) {
		x[i] = (((a * x[i-1]) + b) %% m);
	}
	return(x/m);
}

reverse_base <- function(number, base) {
	n = number;
	result <- vector();
	while (n != 0) {
		result = c(result, n %% base);
		n = n %/% base;
	}
	return(result);
}

gen_Van_der_Corput <- function(number, base) {
	n <- reverse_base(number, base);
	result = 0;
	for (i in 1:length(n)) {
		result = result + (n[i] / (base^i));
	}
	return(result);
}

## Problem - 1 ##

###
v25 <- vector(length = 25);
for (i in 1:25) {
	v25[i] = gen_Van_der_Corput(i, 2);
}

cat(v25, "\n");

rm(v25);

###
v1000 <- vector(length = 1000);
for (i in 1:1000) {
	v1000[i] = gen_Van_der_Corput(i, 2);
}

pdf("pair_vdc.pdf");
plot(v1000[1:999], v1000[2:1000], type = "p", xlab = "x(i)", ylab = "x(i+1)", main = "", pch = '.');

rm(v1000);

###
v100 <- vector(length = 100);
for (i in 1:100) {
	v100[i] = gen_Van_der_Corput(i, 2);
}

v100000 <- vector(length = 100000);
for (i in 1:100000) {
	v100000[i] = gen_Van_der_Corput(i, 2);
}

l100 <- lcg(16807, 0, 2^31 - 1, 1, 100);

l100000 <- lcg(16807, 0, 2^31 - 1, 1, 100000);

pdf("v100.pdf");	hist(v100, breaks = 100, xlab = "X", main = "");
pdf("v100000.pdf");	hist(v100000, breaks = 100, xlab = "X", main = "");
pdf("l100.pdf");	hist(l100, breaks = 100, xlab = "X", main = "");
pdf("l100000.pdf");	hist(l100000, breaks = 100, xlab = "X", main = "");

rm(v100, v100000, l100, l100000, lcg);

## Problem - 2 ##

x100 <- matrix(nrow = 100, ncol = 2);
for (i in 1:100) {
	x100[i,] <- c(gen_Van_der_Corput(i, 2), gen_Van_der_Corput(i, 3));
}

x100000 <- matrix(nrow = 100000, ncol = 2);
for (i in 1:100000) {
	x100000[i,] <- c(gen_Van_der_Corput(i, 2), gen_Van_der_Corput(i, 3));
}

pdf("x100.pdf");	plot(x100, xlab = "phi_2(i)", ylab = "phi_3(i)", main = "", pch = '.');
pdf("x100000.pdf");	plot(x100000, xlab = "phi_2(i)", ylab = "phi_3(i)", main = "", pch = '.');

rm(list = ls());
\end{lstlisting}
\newpage
\noindent{\textbf{Results 1}}\\
The first 25 values of the Van der Corput sequence $x_1, x_2, \cdots, x_{25}$ using the radical inverse function $x_i \:= \phi_{2}(i)$ :\\
0.5 ; 0.25 ; 0.75 ; 0.125 ; 0.625 ; 0.375 ; 0.875 ; 0.0625 ; 0.5625 ; 0.3125 ; 0.8125 ; 0.1875 ; 0.6875 ; 0.4375 ; 0.9375 ; 0.03125 ; 0.53125 ; 0.28125 ; 0.78125 ; 0.15625 ; 0.65625 ; 0.40625 ; 0.90625 ; 0.09375 ; 0.59375

The plot of pairs $(x_i, x_{i+1})$ for 1000 values from the Van der Corput sequence :
\begin{figure}[H]
	\centering
	\includegraphics[width=0.75\textwidth]{"pair_vdc".pdf}
		\caption{Plot $(x_i, x_{i+1})$}
\end{figure}

The points $(x_i, x_{i+1})$ of values generated from the Van der Corput sequence lie in some definite region or hyperplanes (here, parallel lines) in $\mathbb{R}^2$.
\newpage
The comparision of distribution of the Van der Corput and the LCG (a = 16807, b = 0, m = $2^{31}$-1, seed = 1) sequences :
\begin{figure}[H]
	\centering
	\subfloat[Van der Corput]{\includegraphics[width=0.55\textwidth]{"v100".pdf}}
	\subfloat[LCG]{\includegraphics[width=0.55\textwidth]{"l100".pdf}}
		\caption{Comparision for 100 values}
\end{figure}
\begin{figure}[H]
	\centering
	\subfloat[Van der Corput]{\includegraphics[width=0.55\textwidth]{"v100000".pdf}}
	\subfloat[LCG]{\includegraphics[width=0.55\textwidth]{"l100000".pdf}}
		\caption{Comparision for 100000 values}
\end{figure}
\newpage
\noindent{\textbf{Results 2}}\\
The plot of pairs $x_i = (\phi_{2}(i), \phi_{3}(i))$ from the Halton sequence :
\begin{figure}[H]
	\centering
	\subfloat[100 values]{\includegraphics[width=0.55\textwidth]{"x100".pdf}}
	\subfloat[100000 values]{\includegraphics[width=0.55\textwidth]{"x100000".pdf}}
		\caption{Plot $x_i = (\phi_{2}(i), \phi_{3}(i))$}
\end{figure}

The points $x_i = (\phi_{2}(i), \phi_{3}(i))$ of values generated from the Halton sequence are perfectly dispersed in $\mathbb{R}^2$.
\end{document}

%#Made by Ayush Sharma#
%#Signed as AShar#