\documentclass[11pt]{article}
\usepackage{natbib,mybigpackage}
\usepackage{algorithm}
%\usepackage{program}
%\usepackage{algpseudocode}
\usepackage{algorithmic}
\usepackage{listings}

\usepackage{multirow}
%\def\xbf{\mathbf{x}}
%\def\zbf{\mathbf{z}}
%\def\xibf{\mathbf{\xi}}
\title{MA 226 - Assignment Report 11}
\author{Ayush Sharma\\150123046}
\begin{document}
\titlepage
\newpage
%\begin{itemize}
%\item[Question.]
A financial asset, the process $\{S(t)\}$ is a GBM with drift parameter $\mu$, volatility parameter $\sigma$, and initial value $S(0)$ if
$$S(t) = S(0)exp([\mu - \frac{\sigma^{2}}{2}]t + \sigma W(t)),$$
\hspace{10mm}where $\{W(t)\}$ a standard BM.\\
As with the case of a BM, we have a simple recursive procedure to simulate a GBM at $0 = t_0 < t_1 < \cdots t_n$ as
$$S(t_i + 1) = S(t_i) exp([\mu - \frac{\sigma^{2}}{2}](t_{i+1} - t_i) + \sigma \sqrt{t_{i+1} - t_i}Z_{i+1}),$$
\hspace{10mm}where $Z_1, Z_2, \cdots , Z_n$ are independent $\mathcal{N}(0, 1)$ variates.\\
In the interval [0, 5], taking both positive and negative values for $\mu$ and for at least two different values of $\sigma^2$, simulate and plot at least 10 sample paths of the GBM (taking sufficiently large number of sample points for each path).\\
Also, by generating a large number of sample paths, compare the actual and simulated distributions of $S(5)$.
Calculate expectation and variance of $S(5)$ and match it with the theoretical values.\\
\newpage
\noindent{\textbf{Code for R}}
\begin{lstlisting}
sink("output.txt");

set.seed(1);

size = 5000;
n = 10;
m = 1000;
range_time  = 5;
dt = range_time/size;
sdt = sqrt(dt);

mu <- c(0.05, -0.05);
sigma <- c(0.25, 0.3);

S0 = 1;

for (p in 1:2) {
	for (q in 1:2) {
		S <- matrix(1, nrow = (size + 1), ncol = 10);

		for (i in 1:n) {
			for (j in 2:(size + 1)) {
				S[j,i] = S[j - 1,i] * exp(((mu[p] - (sigma[q]^2)/2) * dt) + (sigma[q] * sdt * rnorm(1, mean = 0, sd = 1)));
			}
		}

		pdf(paste("plot",p,q,".pdf"));
		for (i in 1:n) {
			plot(seq(0, 5, dt), S[,i], type = 'l', xlim = c(0,5), ylim = c(-1,5), col = i, verticals = FALSE, do.points = FALSE, main = "", xlab = "", ylab = "")
			par(new = TRUE)
		}
		title(ylab = 'S', xlab = 'Time');
#		legend('topright', legend = c(paste("mu =", mu[p]), paste("sigma =", sigma[q])), lty = 0, col = "white", bty = 'n');

#		cat("\n\nTaking mu", mu[p], "and sigma", sigma[q],", and sample size", n, "::") ;
#		cat("\nSample expectation and variance of S(5) are estimated to be", mean(S[(5/dt + 1),]), ", and", var(S[(5/dt + 1),]), ", respectively.");
#		cat("\nWhile, theoretical expectation and variance of S(5) are", (S0 * exp(mu[p] * 5)), ", and", (S0 * exp(2 * mu[p] * 5) * (exp((sigma[q]^2) * 5) - 1)), ", respectively.");

		s <- vector(length = m);
		for (i in 1:m) {
			s[i] = 1;
			for (j in 2:(size + 1)) {
				s[i] = s[i] * exp(((mu[p] - (sigma[q]^2)/2) * dt) + (sigma[q] * sdt * rnorm(1, mean = 0, sd = 1)));
			}	
		}
		cat("\n\nTaking mu", mu[p], "and sigma", sigma[q],", and sample size", m, "::") ;
		cat("\nSample expectation and variance of S(5) are estimated to be", mean(s), ", and", var(s), ", respectively.");
		cat("\nWhile, theoretical expectation and variance of S(5) are", (S0 * exp(mu[p] * 5)), ", and", (S0 * exp(2 * mu[p] * 5) * (exp((sigma[q]^2) * 5) - 1)), ", respectively.");

	}
}

sink();
\end{lstlisting}
\newpage
\noindent{\textbf{Results}}\\
The plots of the sample paths generated for the Geomeric Brownian Motion ::
\begin{figure}[H]
	\centering
	\subfloat[$\mu = 0.05$ and $\sigma = 0.25$]{\includegraphics[width=0.55\textwidth]{"plot 1 1 ".pdf}}%\hspace{10mm}
	\subfloat[$\mu = 0.05$ and $\sigma = 0.3$]{\includegraphics[width=0.55\textwidth]{"plot 1 2 ".pdf}}\\
	\subfloat[$\mu = -0.05$ and $\sigma = 0.25$]{\includegraphics[width=0.55\textwidth]{"plot 2 1 ".pdf}}%\hspace{10mm}
	\subfloat[$\mu = -0.05$ and $\sigma = 0.3$]{\includegraphics[width=0.55\textwidth]{"plot 2 2 ".pdf}}
		\caption{Plots of sample paths generated for the Geomeric Brownian Motion}	
\end{figure}
\newpage
Comparision of the theoretical and simulated distributions of $S(5)$, i.e. expectations and variances ::
\begin{table}[H]
\begin{center}
	\begin{tabular}{||c | c || c c | c c ||}
	\hline
		\multirow{2}{*}{$\mu$} & \multirow{2}{*}{$\sigma$} & \multicolumn{2}{|c}{Expectation} & \multicolumn{2}{|c||}{Variance}\\
		 & & Sample & Theoretical & Sample & Theoretical\\ 
	\hline
		\multirow{2}{*}{0.05} & 0.25 & 1.280236 & 1.284025 & 0.5402041 & 0.6048135\\
		 & 0.3 & 1.32613 & 1.284025 & 1.094235 & 0.9369884\\
		\multirow{2}{*}{-0.05} & 0.25 & 0.7602637 & 0.7788008 & 0.2082394 & 0.2224985\\
		 & 0.3 & 0.7724962 & 0.7788008 & 0.3620434 & 0.3446988\\
	\hline
	\end{tabular}
	\caption{Expectation and Variance of $S(5)$, taking 1000 samples.}
\end{center}
\end{table}

\end{document}

%#Made by Ayush Sharma#
%#Signed as AShar#